\documentclass[11pt,a4paper]{scrartcl}
\usepackage[margin=2cm]{geometry}
\setlength{\parskip}{5pt}
% Encoding
\usepackage[utf8]{inputenc}
% Bibliography
\usepackage[backend=biber,bibstyle=ieee,citestyle=numeric]{biblatex}
\bibliography{sps_cw1}
\usepackage[hidelinks]{hyperref} % Clickable links
% Header
\usepackage{fancyhdr}
\pagestyle{fancy}
\lhead{COMS21202 CW1}
\rhead{}
%Figures
\usepackage{graphicx}
\graphicspath{ {images/} }
\usepackage{subfig}
%Per page footnote numbering
\usepackage[perpage]{footmisc}
% Maths
\usepackage{amsmath}
\usepackage{bm}

%Code formatting
\usepackage{listings}
\usepackage{xcolor}
\definecolor{code-comment}{rgb}{0,0.6,0}
\definecolor{code-string}{rgb}{0.58,0,0.82}
\lstset{
	frame=tb,
	language=Python,
	keywordstyle=\color{blue},
	stringstyle=\color{code-string},
	commentstyle=\color{code-comment},
	tabsize=4,
	basicstyle={\small\ttfamily},
	showstringspaces=false,
	breaklines=true,
	aboveskip=3mm,
	belowskip=0mm,
}

\title{An Unknown Signal}
\subtitle{Symbols, Patterns and Signals}
\author{Jacob Daniel Halsey}
\date{March 2020}

\begin{document}
	
\maketitle
	
\section{Least Squares Regression}

The least squares calculations have been implemented in the \lstinline|Segment| methods \lstinline|lsr_polynomial| and \lstinline|lsr_fn|.
They both use the matrix formula: \cite{wolfram_ls_poly}

\[\bm{A}=(\bm{X}^{T}\bm{X})^{-1}\bm{X}^{T}\bm{Y}\]

Where in the case of the polynomial regression:

\[
\bm{X} = \begin{bmatrix}
1 & x_{0} & \left (x_{0}  \right )^{2} & .. & \left (x_{0}  \right )^{k} \\ 
1 & x_{1} & \left (x_{1}  \right )^{2} & .. & \left (x_{1}  \right )^{k} \\ 
.. & .. & .. & .. \\ 
1 & x_{k} & \left (x_{k}  \right )^{2} & .. & \left (x_{k}  \right )^{k}
\end{bmatrix}\]

Or in the case of the arbitrary function ($f$) regression:
\[
\bm{X} = 
\begin{bmatrix}
1 & f\left (x_{0}  \right ) \\ 
1 & f\left (x_{1}  \right ) \\ 
.. & .. \\ 
1 & f\left (x_{k}  \right )
\end{bmatrix}
\]

$\bm{Y}$ is a vector containing the $y$ values $y_{0}$ through to $y_{k}$

And the result $\bm{A}$ is the list of coefficients, in the order $a_{0} + a_{1}x + .. + a_{k}x^{k}$ or $a_{0} + a_{1}f(x)$.

\subsection*{Calculating the Error}

The Sum Squared Error (SSE) is a method of measuring how well a fitted function ($f$) fits a dataset of $n$ points, by calculating the difference between the predicted and actual data. \cite{wolfram_ls_fitting}

\[SSE = \sum_{i=0}^{n} ( y_{i} - f(x_{i}) )^{2}\]

This has been implemented in the \lstinline|ss_error| function, which is called with an array of predicted $y$ values, and array of actual $y$ values.

\section{K-Fold Validation}


\section{Testing}


\section{Training Data}


\printbibliography


\end{document}
